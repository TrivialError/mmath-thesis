\chapter{Discussion and Future Work}
In this work, we formalized the approach of using state resets to detect ASA, and showed how asymmetric ASAs can be constructed from symmetric ASAs. A key observation of our work is that many published ASAs are detectable via realistic state reset attacks (such as virtual machine snapshotting) despite having small state.  As such, we encourage future threat models to incorporate this notion of state resets when evaluating the detectability of new ASAs.

We identify several topics that warrant further exploration.

Our type 1 and type~2 asymmetric ASAs have running times that are multiple times longer than the algorithms they subvert. This would make them detectable by a user or other adversary who is able to time execution of the algorithm. Furthermore, all of the published ASAs which are more efficient (in the sense of running times closer to those of the underlying algorithm) are, to our knowledge, detectable in the presence of state resets. An interesting direction for future work is modeling detection of ASAs based on running time and developing ASAs that resist both time-based detection and state reset-based detection, or proving that an ASA satisfying both is impossible. Timing attacks are but one way to model increased detection capabilities on the part of the user; other capabilities and their effects on known ASAs are also of interest.

Further improvements to our ASAs are possible. Our type 1 asymmetric ASA modification is dependent on the ability of the underlying ASA to leak single bits reliably, and this requirement could potentially be removed. Perhaps modifications could also be made on ``non-flexible'' ASAs. Methods to enable key recovery with fewer ciphertexts likely exist, and would be more effective in certain contexts. For example, suppose a user is changing their symmetric key too often for the ASAs in this work to effectively leak the key. Consider an ASA on symmetric encryption that leaks values in 2 stages: first, it leaks a longer encryption of a locally generated shorter key, resulting in a secret key shared with the external subverter; this process would not be interrupted by key changes. Next, it uses this new key to undetectably leak the targeted secret key directly, requiring interception of fewer ciphertexts. This ASA could be formulated as a generic modification to an underlying symmetric ASA, as we did in \autoref{sec:generalize}. Such an ASA could be a type~1 or type~2 asymmetric ASA depending on the exact implementation of the leaks, and, after the initial shared key is established, would recover keys more quickly than the ASAs we presented in this thesis.

Further to the above, it is an open question whether a stateless type~1 asymmetric ASA exists. While we have argued that state does not lead immediately to detection, the effectiveness of a stateful ASA can be mitigated by using state resets. In fact, any ASA that requires state for key recovery can be fully countered by a state reset after every invocation of the subverted algorithm (as mentioned before, this may have significant impacts on performance). We see no straightforward way of modifying our type~1 asymmetric ASA to make it stateless. We encourage work on discovery of a stateless type~1 ASA or on an impossibility result.

We have addressed the topic of countermeasures to ASAs only briefly in this work. As mentioned, several different avenues exist in the literature: deterministic algorithms \cite{C:BelPatRog14,FSE:DegFarPoe15,CCS:BelJaeKan15}; reverse firewalls using re-randomization \cite{CCS:AteMagVen15}; immunization methods \cite{ACNS:AFMV19}, including a split-program methodology for preventing ASAs \cite{CCS:RTYZ17,AC:RTYZ16,CCS:TanYun17}; and so-called self-guarding cryptographic schemes \cite{CSF:FisMaz18}. All of these solutions assume some extra trusted component (for example, a trusted firewall system, a period of time where the scheme is not subverted, or an unsubvertable algorithm composition step). Each solution is able to produce significant guarantees on the scheme's resistance to ASAs. These countermeasures work against our ASAs as well, but we nonetheless encourage more work on methods to prevent ASAs which are simple and easy to implement in practice.
