\chapter{Preliminaries and Definitions} \label{sec:preliminaries}

\section{Games and algorithms}
Proofs in this work will use the cryptographic game-playing framework \cite{EC:BelRog06}. In these games, assignment is denoted by $\leftarrow$, while random sampling is denoted by $\sample$. We write $y \sample \mathsf{A}(x)$ to denote running the probabilistic algorithm $\mathsf{A}$ on input $x$, and assigning the result to $y$. If we wish to specify the random coins $r$ used in a randomized algorithm, we will write $y \leftarrow \mathsf{A}(x;r)$. We will also write $\mathsf{A} \Rightarrow y$ to indicate that $\mathsf{A}$ returns $y$.

We will use min-entropy as a measure of the randomness of an algorithm. Define $\eta_\mathsf{A}$ according to
\shortlongeqn[,]{
2^{-\eta_\mathsf{A}} = \max\limits_{x,y}\left(\prob{y\leftarrow \mathsf{A}(x;r)}\right)
}
where the probability is taken over the choice of coins $r$. The min-entropy of $\mathsf{A}$ is $\eta_\mathsf{A}$. A null value is denoted $\bot$. If $G$ is a game, then $\prob{G}$ indicates the probability that $G$ returns true. 

We will denote game adversaries by script letters (e.g. $\mathcal{A}$). An adversary $\mathcal{A}$ is simply an algorithm. The notation $\mathcal{A}^{\mathcal{O}}$ indicates that the adversary $\mathcal{A}$ has access to the oracle $\mathcal{O}$ for use as a subroutine. The running time of $\mathcal{A}$ is the worst-case execution time of $\mathcal{A}$ including the time it takes to execute any subroutines.

The game-playing framework was nicely formalized by Bellare and Rogaway \cite{EC:BelRog06}. The framework enables more structured methods for cryptographic proofs, as it easily allows for incremental steps and standardized techniques. The primary tool is that of a ``game transition'', where we bound the difference between the probabilities that two games return true. We use many game transitions to create a chain of games between two games (representing two security notions of interest), and obtain a total bound between the success probabilities of the two games at the end of the chain. We will use several common techniques for game transitions throughout this work, and introduce the details as required.


\section{Cryptographic schemes}
A cryptographic scheme $\mathsf{\Lambda}$ is a set of algorithms $\mathsf{\Lambda.Alg}_1,..., \mathsf{\Lambda.Alg}_u$. We will be using several cryptographic schemes in this thesis, including symmetric encryption and public-key encryption. We introduce these here, and other schemes as needed.

A symmetric encryption scheme \textsf{SE} is composed of three algorithms: \textsf{SE.KeyGen}, \textsf{SE.Enc}, and \textsf{SE.Dec}. \textsf{SE.KeyGen} randomly selects a single secret key $k$ of length \textsf{SE.klen} from $\{0,1\}^\mathsf{SE.klen}$. \textsf{SE.Enc} is a randomized algorithm with coins $r\in \{0,1\}^\mathsf{SE.rlen}$ and takes a key and a plaintext $m\in \{0,1\}^\mathsf{SE.mlen}$ and produces a ciphertext $c\in \{0,1\}^\mathsf{SE.clen}$. \textsf{SE.Dec} is a deterministic algorithm, takes a key and a ciphertext, and returns a plaintext or $\bot$, indicating an error.

A public-key (or asymmetric) encryption scheme \textsf{PKE} is similarly composed of three algorithms: \textsf{PKE.KeyGen}, \textsf{PKE.Enc}, and \textsf{PKE.Dec}. \textsf{PKE.KeyGen} randomly generates a secret key $sk$ and a public key $pk$. \textsf{PKE.Enc} is a randomized algorithm with coins $r\in \{0,1\}^\mathsf{PKE.rlen}$ and takes a public key and a plaintext $m\in \{0,1\}^\mathsf{PKE.mlen}$ and produces a ciphertext $c\in \{0,1\}^\mathsf{PKE.clen}$ (note that we will consider only fixed length ciphertexts for public-key schemes). \textsf{PKE.Dec} is a deterministic algorithm, takes a secret key and a ciphertext, and returns a plaintext or $\bot$, indicating an error.

We say that a public-key encryption scheme is $\delta$-correct if, for all $sk,pk$ generated from \textsf{PKE.KeyGen} and $m$,
\shortlongeqn[,]{
\prob{\mathsf{PKE.Dec}(sk, \mathsf{PKE.Enc}(pk, m))=m}\ge \delta
}
where the probability is taken over the choice of coins $r$ for the encryption function. We could define an analogous property for symmetric encryption, but we will mostly assume that such a $\delta$ value will always be 1 unless otherwise stated.

\section{Pseudo-random functions and random oracles}
In this work, we will make use of two standard ways of talking about functions whose output is hard to predict on new inputs. The first is the notion of a pseudo-random function (PRF). Let $\mathsf{F}:K_\mathsf{F}\times \{0,1\}^* \rightarrow W$ be a function, for some output set $W$ and key space $K_\mathsf{F}$. Let the PRF game for \textsf{F} be as defined in \autoref{game:prfgame}. For an adversary $\mathcal{F}$ in the PRF game for \textsf{F}, we define the advantage of $\mathcal{F}$ as
\shortlongeqn[.]{
\text{Adv}^\mathrm{PRF}_\mathsf{F}(\mathcal{F})=\left\lvert \prob{\text{PRF}_\mathsf{F}(\mathcal{F})}-\frac{1}{2} \right\rvert
}

\iffullversion
Note that we can also write the advantage as follows:
\begin{align*}
\text{Adv}^\mathrm{PRF}_\mathsf{F}(\mathcal{F})&=\left\lvert \prob{\text{PRF}_\mathsf{F}(\mathcal{F})}-\frac{1}{2} \right\rvert \\
&=\left\lvert \prob{\mathcal{F}\Rightarrow 1 \text{ and } b=1} + \prob{\mathcal{F}\Rightarrow 0 \text{ and } b=0} -\frac{1}{2} \right\rvert \\
&=\left\lvert \frac{1}{2}\condprob{\mathcal{F}\Rightarrow 1}{b=1} + \frac{1}{2}\condprob{\mathcal{F}\Rightarrow 0}{b=0} -\frac{1}{2} \right\rvert \\
&=\left\lvert \frac{1}{2}\condprob{\mathcal{F}\Rightarrow 1}{b=1} + \frac{1}{2} - \frac{1}{2}\condprob{\mathcal{F}\Rightarrow 1}{b=0} -\frac{1}{2} \right\rvert \\
&=\frac{1}{2} \left\lvert \condprob{\mathcal{F}\Rightarrow 1}{b=1} - \condprob{\mathcal{F}\Rightarrow 1}{b=0} \right\rvert. \\
\end{align*}
We will use the equivalence of the above two forms in a variety of contexts. In particular, for game transitions, it will be useful to recognize this equivalence when the $b=0$ and $b=1$ cases of the PRF game define two other different games that $\mathcal{F}$ plays. Then the difference between the success probability of the two games can be seen with the above relation to be the advantage of $\mathcal{F}$ at the PRF game.
\fi

\begin{figure}
\centering
\begin{pchstack}
\myprocedure[linenumbering]{PRF$_{\mathsf{F}}(\mathcal{F})$}{
	k \sample K_\mathsf{F} \\
	X \leftarrow \emptyset \\
	b \sample \{0,1\} \\
	b' \sample \mathcal{F}^{\mathcal{O}_\mathsf{F}} \\
	\pcreturn b=b'
}
\pchspace
\myprocedure[linenumbering]{$\mathcal{O}_\mathsf{F}(x)$}{
	\pcif b=0 \pcthen w \leftarrow \mathsf{F}(k,x) \\
	\pcif b=1 \pcthen \\
	\pcind \pcif x \notin X \pcthen \\
	\pcind \pcind w_x\sample W \\
	\pcind \pcind X \leftarrow X\cup \{x\} \\
	\pcind \pcreturn w_x
}
\end{pchstack}
\caption[The PRF security game]{The PRF security game for PRF $\mathsf{F}$ and adversary $\mathcal{F}$.}
\label{game:prfgame}
\end{figure}

The second way to talk about functions with unpredictable output is by using the random oracle model. This model is useful for situations in which there is no secret input to the function \textsf{F}, so the game in \autoref{game:prfgame} is no longer relevant. In this model, we replace \textsf{F} by a lazily-sampled random function, and provide oracle access to this function to all game adversaries. A lazily-sampled random function will return random outputs on previously-unseen queries, and outputs consistent with previous outputs for any previously-seen queries. (In the case of $b=1$ in the PRF game, the oracle $\mathcal{O}_\mathsf{F}$ behaves as a lazily-sampled random function.)
